\chapter*{Introduction}
\addcontentsline{toc}{chapter}{Introduction générale}
%\setlength{\parindent}{2cm} 
%\setlength{\parindent}{1cm}
\hspace{0.58cm}
Les séries temporelles sont des données qui sont collectées au fil du temps et qui sont
organisées de manière chronologique. Elles peuvent être utilisées pour analyser et comprendre les tendances et les modèles dans les données au fil du temps, ainsi que pour faire des prévisions sur les événements futurs. Les séries temporelles peuvent être trouvées dans de nombreux domaines, tels que la finance, l’économie, la météorologie, la santé...\\

Les modèles de rupture, également connus sous le nom de modèles de changement de régime, sont des outils statistiques qui permettent de détecter et de modéliser les changements dans la structure d’une série temporelle. Ces modèles sont souvent utilisés pour analyser des données économiques, financières, météorologiques ou sociales, où il peut y avoir des périodes de stabilité suivies de périodes de changement soudain.\\

Les modèles de rupture permettent d’identifier les moments où un changement significatif s’est produit dans les données et de modéliser le comportement de la série temporelle avant et après ce changement. Autrement dit, pour une séquence donnée de variables aléatoires $(X_i)_{1\leq i\leq n}$, nous essayons de trouver un point de rupture $k$ où les éléments $X_1, ..., X_k$ ont une fonction de distribution identique $f_1$ et les éléments $X_{k+1}, ..., X_n$ sont distribuées selon une autre densité de probabilité $f_2$.\\

Dans le rapport qui suit nous allons nous pencher sur le Modèle de Rupture pour la loi normale asymétrique (Skew normal distribution) et l’appliquer aux données mensuelles de TER.\\



